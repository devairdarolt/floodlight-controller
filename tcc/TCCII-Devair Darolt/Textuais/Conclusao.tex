\chapter{Conclusão}
\label{cap:conclusao}
Com o avanço crescente da tecnologia, o volume de dados trafegados para o consumo de aplicações vem se tornando cada vez maior. Com esse aumento houve a necessidade de infraestruturas mais robustas, tanto por parte de provedores de serviços utilizando nuvens ou infraestruturas de clusters, quanto por parte das infraestruturas consumidoras, como redes de universidades, empresas, órgãos governamentais, entre outros. Considerando essa demanda, as Redes Definidas por Software (SDN) surgiram como uma proposta, que utiliza da estratégia de um plano de controle separado para efetuar tarefas de otimização dos tráfegos, segurança, tolerância a falhas e escalabilidade. 

Para a realização do trabalho foi efetuado um estudo geral sobre as SDN no capítulo \ref{cap:revisao}, fazendo uma revisão de literatura sobre a arquitetura, separada em três principais planos: plano de gerenciamento, plano de controle e plano de dados. 

No plano de gerenciamento é que se encontra a maioria das aplicações que utilizam os dados fornecidos pelo plano de controle. Controle de acesso, \emph{loadbalancer}, \textit{interface} de usuário são aplicações comumente encontradas neste plano. A revisão do plano de controle, fez um levantamento entre os duas principais categorias de controladores: os distribuídos, criados para dar maior tolerância a falhas e gerenciar redes cuja as infraestruturas são geograficamente distantes; e os centralizados, que o presente trabalho teve maior ênfase com o controlador \emph{floodlight}. Uma camada muito importante do plano de controle é a \textit{interface} \emph{northbound}, normalmente utilizando o padrão REST esta tem o objetivo de entregar as funcionalidades do controle da rede para ser acessados por URLs, facilitando na diversificação das aplicações (\textit{e.g.}, python, javascript, c\#).

O plano de dados é dividido em duas camadas principais, a primeira contém os dispositivos da infraestrutura, comutadores e computadores. Ainda neste plano podemos dar destaque a equipamentos emulados e simulados, que permitem que estudos possam ser executados sem o custo elevado de compras dos equipamentos. A segunda camada do plano de dados é composta pelos protocolos e padrões definidos para haver comunicação com o plano de controle. Nesta camada o trabalho deu ênfase ao protocolo OpenFlow 1.3 \emph{Long-Term Support} (LTS) por ser amplamente utilizado no mercado.

Diferente das redes convencionais, o SDN utiliza fluxos de pacotes. Estes são sequencias de pacotes que correspondem a regras inseridas nos comutadores da rede pelo controlador. Para a tarefa de gerenciamento dos fluxos o trabalho deu ênfase ao controlador \emph{Floodlight}, detalhado na seção \ref{sec:floodlight}. Em sua arquitetura modular \textit{open source}, desenvolvida na linguagem de programação Java este controlador foi utilizado para adição dos módulos CMCR, CMCP, RR e CMT. 

Para criação da infraestrutura foi utilizado no trabalho o emulador \textit{Mininet}. Ferramenta altamente utilizada para modelagem e pesquisas das SDN baseadas no protocolo \textit{OpenFlow}, por ser gratuita permitindo várias instancias executadas em diversos computadores interconectados por uma rede normal. Emulando uma rede SDN bem próxima da realidade.

Com o ambiente gerado foi utilizado para o estudo quatro estratégias distintas para o controle dos fluxos: \textit{Round-Robin} (RR), caminho mais curto reativo (CMCR), caminho mais curto proativo (CMCP) e caminho de menor tráfego (CMT). A política de RR utiliza a vantagem de múltiplos caminhos para distribuir o tráfego de pacotes em caminhos alternativos utilizando uma fila circular. A política de CMCR utiliza sempre o primeiro caminho de uma lista de menor quantidade de saltos para formar o caminho mais curto, a cada salto o comutador precisa consultar o controlador que decide gerando regras necessárias para que o pacote chegue ao próximo salto, até que todos os comutadores do caminho sejam configurados. A política de CMCP possui essa mesma estratégia da menor quantidade de saltos, porém com a diferença que nessa, o controlador configura previamente todos os comutadores do caminho, reduzindo o número de consultas. A política de CMT usa as informações fornecidas pelo protocolo Openflow de cada comutador para calcular a largura de banda utilizada por cada porta, calculando assim a taxa de tráfego de cada enlace. Com isso permite que o controlador escolha o caminho com menor tráfego para o fluxo.

Os resultados foram obtidos utilizando o NAS, programa de avaliação de desempenho de super computadores e infraestruturas de rede. As aplicações do NAS foram desenvolvidas com MPI e cada uma possui seu próprio padrão de comunicação e processamento, permitindo que o comportamento das políticas possam ser estudadas na utilização de uma aplicação real.

Em suma, os estudos apontam que quando uma aplicação com pouca comunicação interna é executada, as políticas que utilizam o caminho com menor número de saltos se saem melhor, ainda que utilizem sempre os mesmos caminhos, já que estes não geram um tráfego suficiente para ocasionar atrasos na aplicação. As aplicações com maior volume de dados transmitidos (\textit{e.g.}, CG, SP, MG)  possuem um desempenho  melhor nas políticas RR e CMT vela vantagem dos múltiplos caminhos. 

\section{Trabalhos futuros}

Utilizar os recursos oferecidos pelos dispositivos de encaminhamento da rede para melhorar a gerenciamento do tráfego é uma tarefa que vem sendo explorada ao longo dos anos, a leitura dessas informações ocasionam dois problemas, na primeira: para obter maior precisão o controlador envia pacotes com mais frequência causando sobrecarregando o canal seguro, ou o controlador; no segundo envia com menos frequência e obtém estatísticas com baixa precisão. Em~\cite{singh2017estimation}, é reforçado a necessidade de aplicações  desenvolvidas de modo a permitir uma melhor leitura de estatísticas de fluxos e portas. A aplicação \textit{sFLow} por exemplo, permite com que comutadores \textit{OpenFlow} mandem periodicamente uniformemente em taxas ajustáveis, cabeçalhos de pacote para o monitoramento dos fluxos, de modo que não sobrecarregue o controlador ~\cite{mogul2010devoflow}. Gerando eventos a cada 1/1000 pacotes a ferramenta recebe as informações utilizadas para medir a utilização das portas. Outra ferramenta criada para resolver este problema é o \textit{NetFlow}, desenvolvida pela empresa CISCO tem o objetivo de colher as estatísticas sem sobrecarregar o controlador. Com essa perspectiva trabalhos futuros utilizando uma melhor leitura para a seleção de rotas podem ser usadas para determinar ganhos mais significativos no gerenciamento do plano de dados. 

Ainda em relação da otimização do tráfego nas SDNs, pesquisas como \textit{SDPredictNet-A}, utiliza de Redes Neurais Artificiais, treinadas para prever eventos e encontrar o caminho ótimo para os fluxos \cite{sanagavarapu2021sdpredictnet}. Em \cite{casas2020intelligent}, o autor usa aprendizagem por reforço criando um plano acima do plano de gerenciamento. Essas técnicas utilizadas em conjunto com ferramentas eficientes de coleta de estatísticas podem proporcionar uma aproximação mais exata da imagem que o controlador cria em \textit{cache} para representar o plano de dados, podendo proporcionar trabalhos a serem desenvolvidos.

Outras pesquisas relacionadas a engenharia de tráfego, buscam reduzir o consumo de energia. Consolidar os caminhos e desabilitar temporariamente dispositivos comutadores ociosos sem violar o SLA,  podem reduzir o custo em SDN de grande porte \cite{torkzadeh2021energy}. 





