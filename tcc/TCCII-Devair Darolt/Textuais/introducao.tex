\chapter{Introdução}
\label{cap:introducao}

O gerenciamento de redes convencionais baseadas no \emph{Internet Protocol} (IP) é um desafio para os administradores de redes.
Para configurar diretrizes de segurança e encaminhamento de pacotes, é necessário utilizar instruções de baixo nível especificadas pelos fornecedores dos dispositivos. 
Tradicionalmente, cada dispositivo de rede possui uma camada de dados e uma camada de controle implementadas como parte do equipamento, tornando essas infraestruturas rígidas, com pouca possibilidade de inovação~\cite{kim2013}. Redes com alta complexidade de formulação e manutenção geralmente necessitam de intervenções manuais significativas em sua gestão, estando mais propensas a falhas e por isso são mais difíceis de serem atualizadas e gerenciadas~\cite{benson2009}.

O conceito de Redes Definidas por Software (\textit{Software-Defined Networking} -- SDN), primeiramente desenvolvido como um experimento acadêmico, tornou-se uma tecnologia emergente para evolução das redes de computadores.
Em suma, SDN introduziu a separação entre o plano de controle e o plano de dados, transformando os componentes de comunicação em simples dispositivos de encaminhamento. 
Neste cenário, a configuração das políticas de encaminhamento e balanceamento de carga pode ser implementada sobre um controlador logicamente centralizado, denominado \emph{Network Operation System} (NOS) ~\cite{kreutz2015}.

A separação dos planos nos dispositivos de encaminhamento oferece uma série de benefícios para pesquisadores e administradores: \textit{(i)}~Ambiente simplificado e menos propenso a erros ocasionados por modificações em políticas. Ainda, as ações de configuração são realizadas por interfaces pré-definidas que abstraem os recursos reais. \textit{(ii)}~Algoritmos de controle podem automaticamente reagir a mudanças no estado da rede e  manter as políticas de alto nível intactas. Assim, alterações podem ser realizadas nas tabelas de fluxo dos dispositivos de encaminhamento sem impactar a aplicação final. \textit{(iii)}~A centralização da lógica de controle em um controlador com conhecimento global do estado da rede simplifica o desenvolvimento de funcionalidades de gerenciamento. \textit{(iv)}~Redução de ~\textit{vendor lock-in} possibilita maior integração de equipamentos de diferentes fabricantes.

Em SDN, os dispositivos de encaminhamento realizam operações elementares, baseadas em regras para determinar ações sobre os pacotes recebidos. 
Essas instruções são instaladas nos dispositivos pela interface \emph{southbound}\footnote[1]{~\textit{API} responsável pela comunicação com os dispositivos de encaminhamento de forma padronizada} (e.g., OpenFlow, ForCES).
Nesse cenário, o plano de dados é representado pelos dispositivos de encaminhamentos que são interconectados entre si formando uma infraestrutura SDN. 
Já o plano de controle contém as camadas responsáveis pelas informações da infraestrutura de rede e possui uma série de elementos que programam o comportamento dos dispositivos de encaminhamento. Toda lógica de controle reside nesse plano. A interface \emph{northbound} realiza a abstração das instruções de baixo nível utilizadas pela interface \emph{southbound} e as oferecem como serviços ao plano de gerenciamento. O plano de gerenciamento contém o conjunto de aplicações que aproveitam as funcionalidades fornecidas pela interface \emph{northbound} para fazer o controle lógico da rede~\cite{kreutz2015}, tal como encaminhamento, ~\emph{firewall}, balanceamento de carga e monitoramento.

%O plano de controle pode ser implementado de maneira centralizada ou distribuída~\cite{guedes:2}.
%Na centralizada, um único controlador é capaz de gerenciar toda a infraestrutura da rede, entretanto, estando propenso a falhas que inviabilizariam a utilização da arquitetura. Embora centralizado, esse modelo é capaz de atender redes de menor porte como \textit{datacenters} de tamanho médio e redes privadas.
%Por sua vez, uma SDN que possua um sistema distribuído de controle é mais tolerante a falhas e pode ser escalada para redes de alta densidade de dispositivos. 
%Em nuvens computacionais, com vários \textit{datacenters} geograficamente distantes, é utilizado o plano de controle híbrido e hierárquico, ou seja, uma combinação de soluções centralizadas e distribuídas. 

Diferente das redes convencionais, SDN trabalha com políticas de encaminhamento de fluxos ao invés de pacotes. Um fluxo é uma sequência de pacotes que possuem características semelhantes, tais como endereços de origem, destino, identificadores de \emph{Virtual Local Area Network} (VLAN) , protocolo, número da porta de origem e destino. 
As informações são combinadas compondo uma tupla que representa o fluxo em questão.
Assim, os algoritmos de encaminhamento são aplicações executadas no plano de controle que utilizam essas tuplas com o objetivo de configurar os dispositivos SDN, para que estes realizem operações elementares de encaminhamento com base em regras de suas tabelas de fluxos, de modo a estabelecer a comunicação entre origem e destino. 

%///////////////////////////////////////////////////////////////////////////////////
\section{Objetivo}
\label{sec:objetivo}
Diante da motivação apresentada, o presente trabalho investiga o comportamento de algoritmos de encaminho de fluxos em SDN, realizando uma análise experimental comparativa. Para que o objetivo proposto neste trabalho seja realizado, os seguintes objetivos específicos devem ser atingidos:
\begin{itemize}
    \item Investigar o paradigma SDN;
    \item Revisar algoritmos de encaminhamento existentes e seus objetivos;
    \item Estudar o funcionamento do emulador  \textit{Mininet} e do controlador \textit{Floodlight};
    \item Implementar os algoritmos de encaminhamento no controlador \textit{Floodlight};
    \item Definir as métricas para representar o desempenho da rede;
    \item Coletar dados experimentais; e
    \item Realizar uma análise dos resultados obtidos.
    
\end{itemize}
%////////////////////////////////////////////////////////////////////////////////////
% \section{Metodologia}
% \label{sec:metodologia}
% % Para alcançar os objetivos propostos, este trabalho será desenvolvido em 8 (oito) etapas principais. Inicialmente será realizado um estudo sobre as arquiteturas dos SDNs. Em um segundo momento, uma revisão sobre algoritmos de grafos com controle de fluxo. A terceira etapa consiste em realizar um estudo sobre o funcionamento do controlador lógico SDN Floodlight.
% % A quarta etapa abordará o estudo das politicas de controle de fluxo no Floodlight.
% % A quinta etapa corresponde a implementação dos algoritmos de encaminhamento.
% % A sexta etapa consiste em um estudo sobre o comportamento da infraestrutura virtual gerada na ferramenta Mininet.
% % Por fim, uma validação experimental sera feita em forma de um comparativo entre tais rotinas de controle de fluxo.

% Dessa forma, o processo de desenvolvimento deste trabalho envolverá as seguintes etapas:

% \begin{itemize}
%     \item TCC 1:
%     \begin{enumerate}
%        	\item Revisão de literatura sobre a arquitetura SDN e protocolo OpenFlow;
% 		\item Revisão de literatura sobre algoritmos de controle de fluxos e sua representação em grafos;
% 		\item Revisão de literatura sobre arquitetura do controlador Floodlight;
% 		\item Estudo sobre implementação das políticas de encaminhamento de fluxos no controlador Floodlight;
% 		\item Definição dos algoritmos que serão comparados;
% 		\item Escrita do TCC-I;
%     \end{enumerate}
    
%     \item TCC 2:
%     \begin{enumerate}
%         \setcounter{enumi}{6}
%         \item Estudo sobre o emulador Mininet;
% 		\item Implementação dos algoritmos selecionados;
% 		\item Definição das métricas de comparação;
% 		\item Análise experimental;
% 		\item Escrita do TCC-II;
%     \end{enumerate}
% \end{itemize}

%////////////////////////////////////////////////////////////////////////////////////
\section{Estrutura do Trabalho}
\label{sec:estrutura}
O texto está estruturado da seguinte forma: O Capítulo ~\ref{cap:revisao} revisa os conceitos e tecnologias relacionadas com SDN. São revisadas as definições de fluxos de pacotes, OpenFlow, SDN, entre outros elementos necessários para contextualização do trabalho. 
O Capítulo \ref{cap:algoritimos} revisa de forma detalhada a estratégia de funcionamento das políticas de controle de fluxos escolhidas para o presente estudo.
O Capítulo~\ref{cap:protocoleo_experimental} faz uma revisão sobre o emulador utilizado para a simulação, apresentando junto a topologia da rede de múltiplos caminhos criada para o estudo das políticas. Este capítulo faz também uma breve introdução ao \textit{software} para avaliação de desempenho de infraestruturas de comunicação.
O capítulo \ref{cap:analise_resultados} busca apresentar os resultados obtidos pelas aplicações da ferramenta de avaliação utilizada para os testes.
O capítulo \ref{cap:conclusao} faz uma conclusão sobre o trabalho apontando possíveis desenvolvimentos futuros.

